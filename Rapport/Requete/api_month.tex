\paragraph{}
	Renvoie une liste de JSON contenant toute les sessions du mois.

\subsection{Requête}
	\subsubsection{paramètre du path}
		\begin{center}
			\begin{tabularx}{\textwidth}{| T{0.2122\textwidth} | S{0.22\textwidth} | S{0.15\textwidth} | S{0.3\textwidth} |}
				\hline
				Propriété & Type & Obligatoire & Description \\
				\hline
				Month & int & O & numéro du mois sélectionné. \\
				\hline
				Year & int & O & Année sélectionnée. \\
				\hline
			\end{tabularx}
		\end{center}
		
	\subsubsection{Content type}
		\paragraph{}
			text/plain
			
	\subsubsection{Body}
		\paragraph{}
			Pas de body pour la requête.

\subsection{Réponse}
	\subsubsection{Réussite}
		\paragraph{}
			Retourne code 200
			
	\subsubsection{Content-type}
		\paragraph{}
			application/json
	
	\subsubsection{Body}
		\begin{center}
			\begin{tabularx}{\textwidth}{| T{0.2122\textwidth} | S{0.22\textwidth} | S{0.15\textwidth} | S{0.3\textwidth} |}
				\hline
				Propriété & Type & Obligatoire & Description \\
				\hline
				id & int & O & Id de la session. \\
				\hline
				Date & DateTime & O & Date de la session. \\
				\hline
				Time & DateTime & O & Heure de la session. \\
				\hline
				Bike & int & O & Nombre de vélos disponibles pour la session. \\
				\hline
				Cancel & Boolean & O & Etat de la session. \\
				\hline
				idInscription & List & O & Liste des participants. \\
				\hline
				idPerson & Person & O & Identité du participant. \\
				\hline
				lastName & String & O & Nom du participant. \\
				\hline
				firstName & String & O & Prénom du participant. \\
				\hline
			\end{tabularx}
		\end{center}
		
	\newpage
		\paragraph{Exemple JSON}
			\paragraph{}
			\begin{lstlisting}[language=json,firstnumber=1]
{
  id: 15,
  Date: "2019/12/23 00:00",
  time: "1970/01/01 19:00",
  bike: 8,
  Cancel: false,
  idInscription: [{
  	0: {
  	  id: 45,
		idSession: 15,
		  idPerson: {
			  id: 49,
			  lastName: "Isabelle",
			  firstName: "Smits"
			}
		},
	}]
}
\end{lstlisting}
			
	\subsubsection{Code}
		\paragraph{}
			\Href{https://github.com/victorsmits/Aquabike/blob/master/backend/src/Controller/API/MonthControllerApi.php}{MonthControllerApi.php}