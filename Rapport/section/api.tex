\section{Tableau recapitulatif}
		\begin{center}
			\begin{tabularx}{1.11\textwidth}{| T{0.11\linewidth} | S{0.22\linewidth} | S{0.3\linewidth} | S{0.15\linewidth} | S{0.177\linewidth} |}
				\hline
					\footnotesize{Méthode HTTP} & \centering{Requête} & Contrôleur & \footnotesize{Méthode} & Action \\
				\hline
					GET & /api & HomeControllerAPI & index & renvoie les informations des sessions d'aujourd'hui. \\
				\hline
					GET & /api/month /\{month?\}/\{year?\} & MonthControllerAPI & index & renvoie la liste des sessions du mois. \\
				\hline
					GET & /api/profile/ \{username?\} & ProfileControllerAPI & index & Renvoie le profile de l'utilisateur. \\
				\hline
					GET & /api/admin /abonnement & AbonnementControllerAPI & index & renvoie la liste des utilisateurs. \\
				\hline
					GET & /api/TypeSession & RegistrationControllerApi & getTypeSession & Renvoie la liste des types de session. \\
				\hline
			
				\hline
			
				\hline
					POST & /api/admin /renewAbo & AbonnementControllerAPI & RenewAbo & renouvelle l'abonnement de l'utilisateurs et l'inscrit à ses sessions. \\
				\hline
					POST & /api/admin/editAbo & AbonnementControllerAPI & editAbo & Change l'abonnement de l'utilisateur. \\
				\hline
					POST & /api/admin/Cancel & SessionAdministration- ControllerApi & cancel- Session & Annule la session et désinscrit les utilisateurs. \\
				\hline
					POST & /api/admin/recreate & SessionAdministration- ControllerApi & recreate- Session & Recrée la session. \\
				\hline
					POST & /api/admin /autocreate & CreateSessionControllerApi & createSession & Génère des sessions pour x années.\\
				\hline
					POST & /api/login & SecurityControllerApi & login & Authentifie l'utilisateur. \\
				\hline
			\end{tabularx}
		\end{center}
		
		\newpage
			\begin{center}
				\begin{tabularx}{1.11\textwidth}{| T{0.11\linewidth} | S{0.22\linewidth} | S{0.3\linewidth} | S{0.151\linewidth} | S{0.175\linewidth} |}
					\hline
						POST & /api/editProfile & ProfileControllerApi & editProfile & Modifie les données de l'utilisateurs. \\
					\hline
						POST & /api/TypeSession & RegistrationControllerApi & editTypeSession & Modifie les données du type de session. \\
					\hline
					
					\hline
					
					\hline
						DELETE & /api/admin/session /\{id?\} & SessionAdministration- ControllerApi & deleteSession & Supprime la session. \\
					\hline
						DELETE & /api/Desinscription /\{username?\}/\{id?\} & MonthControllerAPI & remove- Inscription & Désinscrit l'utilisateur de la session. \\
					\hline
						DELETE & /api/TypeSession /\{id?\} & RegistrationControllerApi & Delete- TypeSession & Supprime le type de session. \\
					\hline
						DELETE & /api/admin/user /\{id?\} & AbonnementControllerApi & delAbo & Supprime l'utilisateur et ses références. \\
					\hline
			
					\hline
			
					\hline
						PUT & /api/admin/session & CreateSessionControllerApi & index & Créer une nouvelle session. \\
					\hline
						PUT & /api/Inscription & MonthControllerAPI & create- Inscription & Inscrit l'utilisateur à la session. \\
					\hline
						PUT & /api/TypeSession & RegistrationControllerApi & setType- Session & Créer un nouveau type de session. \\
					\hline
						PUT & /api/register & RegistrationControllerApi & register & Créer un nouvelle utilisateur et l'inscrit a ses sessions. \\
					\hline
				\end{tabularx}
			\end{center}


\newpage
\section{/api}
	\paragraph{}
	Renvoie un JSON contenant toute les sessions du jours ainsi que les insciptions.

\subsection{Requête}
	\subsubsection{paramètre du path}
		\paragraph{}
			Pas de Paramètre pour le path.
	
	\subsubsection{Content type}
		\paragraph{}
			text/plain
			
	\subsubsection{Body}
		\paragraph{}
			Pas de body pour la requête.

\subsection{Réponse}
	\subsubsection{Réussite}
		\paragraph{}
			Retourne code 200
			
	\subsubsection{Content-type}
		\paragraph{}
			application/json
	
	\subsubsection{Body}
		\begin{center}
			\begin{tabularx}{\textwidth}{| T{0.2122\textwidth} | S{0.22\textwidth} | S{0.15\textwidth} | S{0.3\textwidth} |}
				\hline
				Propriété & Type & Obligatoire & Description \\
				\hline
				id & int & O & Id de la session. \\
				\hline
				Date & DateTime & O & Date de la session. \\
				\hline
				Time & DateTime & O & Heure de la session. \\
				\hline
				Bike & int & O & Nombre de vélo disponible pour la session. \\
				\hline
				Cancel & Boolean & O & Etats de la session. \\
				\hline
				IdTypeSession & int & O & id du type de session correspondant. \\
				\hline
				idInscription & List & O & Liste des participants. \\
				\hline
				idPerson & Person & O & Identité du participant. \\
				\hline
				lastName & String & O & Nom du participant. \\
				\hline
				firstName & String & O & Prénom du participant. \\
				\hline
			\end{tabularx}
		\end{center}
		
		\newpage
		\paragraph{Exemple JSON}
			\paragraph{}
			\begin{lstlisting}[language=json,firstnumber=1]
{
  id: 15,
  Date: "2019/12/23 00:00",
  time: "1970/01/01 19:00",
  bike: 8,
  Cancel: false,
  IdTypeSession: 1,
  idInscription: [{
  	0: {
  	  id: 45,
		idSession: 15,
		  idPerson: {
			  id: 49,
			  lastName: "Isabelle",
			  firstName: "Smits"
			}
		},
	}]
}
\end{lstlisting}
			
			
	\subsubsection{Code}
		\paragraph{}
			\Href{https://github.com/victorsmits/Aquabike/blob/master/backend/src/Controller/API/HomeControllerApi.php}{HomeControllerApi.php}


\vspace{\baselineskip}
\section{/api/month}
	\paragraph{}
	Renvoie une liste de JSON contenant toute les sessions du mois.

\subsection{Requête}
	\subsubsection{paramètre du path}
		\begin{center}
			\begin{tabularx}{\textwidth}{| T{0.2122\textwidth} | S{0.22\textwidth} | S{0.15\textwidth} | S{0.3\textwidth} |}
				\hline
				Propriété & Type & Obligatoire & Description \\
				\hline
				Month & int & O & numéro du mois sélectionné. \\
				\hline
				Year & int & O & Année sélectionné. \\
				\hline
			\end{tabularx}
		\end{center}
		
	\subsubsection{Content type}
		\paragraph{}
			text/plain
			
	\subsubsection{Body}
		\paragraph{}
			Pas de body pour la requête.

\subsection{Réponse}
	\subsubsection{Réussite}
		\paragraph{}
			Retourne code 200
			
	\subsubsection{Content-type}
		\paragraph{}
			application/json
	
	\subsubsection{Body}
		\begin{center}
			\begin{tabularx}{\textwidth}{| T{0.2122\textwidth} | S{0.22\textwidth} | S{0.15\textwidth} | S{0.3\textwidth} |}
				\hline
				Propriété & Type & Obligatoire & Description \\
				\hline
				id & int & O & Id de la session. \\
				\hline
				Date & DateTime & O & Date de la session. \\
				\hline
				Time & DateTime & O & Heure de la session. \\
				\hline
				Bike & int & O & Nombre de vélo disponible pour la session. \\
				\hline
				Cancel & Boolean & O & Etats de la session. \\
				\hline
				idInscription & List & O & Liste des participants. \\
				\hline
				idPerson & Person & O & Identité du participant. \\
				\hline
				lastName & String & O & Nom du participant. \\
				\hline
				firstName & String & O & Prénom du participant. \\
				\hline
			\end{tabularx}
		\end{center}
		
	\newpage
		\paragraph{Exemple JSON}
			\paragraph{}
			\begin{lstlisting}[language=json,firstnumber=1]
{
  id: 15,
  Date: "2019/12/23 00:00",
  time: "1970/01/01 19:00",
  bike: 8,
  Cancel: false,
  idInscription: [{
  	0: {
  	  id: 45,
		idSession: 15,
		  idPerson: {
			  id: 49,
			  lastName: "Isabelle",
			  firstName: "Smits"
			}
		},
	}]
}
\end{lstlisting}
			
	\subsubsection{Code}
		\paragraph{}
			\href{https://github.com/victorsmits/Aquabike/blob/master/backend/src/Controller/API/MonthControllerApi.php}{MonthControllerApi.php}
	

\vspace{\baselineskip}
\section{/api/profile}
	\paragraph{}
	Renvoie un de JSON contenant toutes les informations de l'utilisateurs

\subsection{Requête}
	\subsubsection{paramètre du path}
		\begin{center}
			\begin{tabularx}{\textwidth}{| T{0.2122\textwidth} | S{0.22\textwidth} | S{0.15\textwidth} | S{0.3\textwidth} |}
				\hline
				Propriété & Type & Obligatoire & Description \\
				\hline
				Username & String & O & Nom d'utilisateur à afficher. \\
				\hline
			\end{tabularx}
		\end{center}
		
	\subsubsection{Content type}
		\paragraph{}
			text/plain
			
	\subsubsection{Body}
		\paragraph{}
			Pas de body pour la requête.

\newpage
\subsection{Réponse}
	\subsubsection{Réussite}
		\paragraph{}
			Retourne code 200
			
	\subsubsection{Content-type}
		\paragraph{}
			application/json
	
	\subsubsection{Body}
		\begin{center}
			\begin{tabularx}{\textwidth}{| T{0.2122\textwidth} | S{0.22\textwidth} | S{0.15\textwidth} | S{0.3\textwidth} |}
				\hline
				Propriété & Type & Obligatoire & Description \\
				\hline
				id & int & O & Id de l'utilisateur. \\
				\hline
				lastName & String & O & Nom du participant. \\
				\hline
				firstName & String & O & Prénom du participant. \\
				\hline
				Abonnement & int & O & Nombre de sessions restant dans l'abonnement. \\
				\hline
				UserName & String & O & Nom d'utilisateur. \\
				\hline
				Email & String & O & Email de l'utilisateur. \\
				\hline
				AboType & int & O & Type d'abonnement de l'utilisateur. \\
				\hline
				Role & String & O & Role donnant les accès a l'utilisateur. \\ 
				\hline
				idInscription & List<Session> & O & Liste des sessions auxquelles l'utilisateur est inscrit. \\
				\hline
				
				\hline
				
				\hline
				id & int & O & Id de la session. \\
				\hline
				Date & DateTime & O & Date de la session. \\
				\hline
				Time & DateTime & O & Heure de la session. \\
				\hline
				Bike & int & O & Nombre de vélos disponibles pour la session. \\
				\hline
				Cancel & Boolean & O & Etat de la session. \\
				\hline

			\end{tabularx}
		\end{center}
		
	\newpage
		\paragraph{Exemple JSON}
			\paragraph{}
			\begin{lstlisting}[language=json,firstnumber=1]
{
  id: 49,
  LastName: "Isabelle",
  FirstName: "Smits",
  Abonnement: 0,
  role: "ROLE_USER",
  Username: "isa",
  Email: "isa_smi@hotmail.com",
  AboType: 20,
  idInscription: [{
  	0: {
      id: 45,
      idSession: {
        id: 15,
	    date: "2019/12/23 00:00",
	    time: "1970/01/01 19:00",
	    bike: 8,
	    cancel: false
	  }
    }
  }]
}
\end{lstlisting}
			
			
	\subsubsection{Code}
		\paragraph{}
			\Href{https://github.com/victorsmits/Aquabike/blob/master/backend/src/Controller/API/ProfileControllerApi.php}{ProfileControllerApi.php}
	

\newpage
\section{/api/admin/abonnement}
	\paragraph{}
	Renvoie une liste de JSON contenant toute les informations des l'utilisateurs.

\subsection{Requête}
	\subsubsection{paramètre du path}
		\paragraph{}
			Pas de Paramètre pour le path.
	
	\subsubsection{Content type}
		\paragraph{}
			text/plain
			
	\subsubsection{Body}
		\paragraph{}
			Pas de body pour la requête.

\subsection{Réponse}
	\subsubsection{Réussite}
		\paragraph{}
			Retourne code 200
			
	\subsubsection{Content-type}
		\paragraph{}
			application/json
	
	\subsubsection{Body}
		\begin{center}
			\begin{tabularx}{\textwidth}{| T{0.2122\textwidth} | S{0.22\textwidth} | S{0.15\textwidth} | S{0.3\textwidth} |}
				\hline
				Propriété & Type & Obligatoire & Description \\
				\hline
				id & int & O & Id de l'utilisateur. \\
				\hline
				lastName & String & O & Nom du participant. \\
				\hline
				firstName & String & O & Prénom du participant. \\
				\hline
				Abonnement & int & O & Nombre de sessions restant dans l'abonnement. \\
				\hline
				UserName & String & O & Nom d'utilisateur. \\
				\hline
				Email & String & O & Email de l'utilisateur. \\
				\hline
				AboType & int & O & Type d'abonnement de l'utilisateur. \\
				\hline
				Role & String & O & Role donnant les accès a l'utilisateur. \\ 
				\hline
				idTypeSession & List<LienPerson- TypeSession> & O & Liste des sessions auxquelles l'utilisateur est inscrit. \\
				\hline
				
				\hline
				
				\hline
				id & int & O & Id du lien entre la personne et le type de session. \\ 
				\hline
				idPerson & int & O & Id de la personne. \\
				\hline
				IdTypeSession & JSON<TypeSession> & O & Type de session. \\
				\hline
				
				\hline
				
				\hline
				id & int & O & Id du type de session. \\
				\hline
				day & String & O & Jour du type de session. \\
				\hline
				Time & DateTime & O & Heure du type de session. \\
				\hline

			\end{tabularx}
		\end{center}
		
	\newpage
		\paragraph{Exemple JSON}
			\paragraph{}
			\begin{lstlisting}[language=json,firstnumber=1]
{
  id: 49,
  LastName: "Isabelle",
  FirstName: "Smits",
  Abonnement: 0,
  role: "ROLE_USER",
  Username: "isa",
  Email: "isa_smi@hotmail.com",
  AboType: 20,
  idTypeSession: [{
  	0: {
	  id: 55
	  IdPerson: 1
	  IdTypeSession: {
	    id: 2
		day: "Mon"
		time: "1970/01/01 20:10"
	  }
    }
  }]
}
\end{lstlisting}
			
			
	\subsubsection{Code}
		\paragraph{}
			\Href{https://github.com/victorsmits/Aquabike/blob/master/backend/src/Controller/API/AbonnementControllerApi.php}{AbonnementControllerApi.php}
	
	
\newpage
\section{/api/TypeSession}
	\paragraph{}
	Renvoie une liste de JSON contenant toute les types de sessions.

\subsection{Requête}
	\subsubsection{paramètre du path}
		\paragraph{}
			Pas de Paramètre pour le path.
	
	\subsubsection{Content type}
		\paragraph{}
			text/plain
			
	\subsubsection{Body}
		\paragraph{}
			Pas de body pour la requête.

\vspace{\baselineskip}
\subsection{Réponse}
	\subsubsection{Réussite}
		\paragraph{}
			Retourne code 200
			
	\subsubsection{Content-type}
		\paragraph{}
			application/json
	
	\subsubsection{Body}
		\begin{center}
			\begin{tabularx}{\textwidth}{| T{0.2122\textwidth} | S{0.22\textwidth} | S{0.15\textwidth} | S{0.3\textwidth} |}
				\hline
				Propriété & Type & Obligatoire & Description \\
				\hline
				id & int & O & Id du type de session. \\
				\hline
				day & String & O & Jour du type de session. \\
				\hline
				Time & DateTime & O & Heure du type de session. \\
				\hline
				idTypeSession & List<LienPerson- TypeSession> & O & Liste des session auquel l'utilisateur est inscrit. \\
				\hline
				
				\hline
				
				\hline
				id & int & O & Id du lien entre la person et le type de session. \\ 
				\hline
				idPerson & int & O & Id de la person. \\
				\hline
				
				\hline
				
				\hline
				id & int & O & Id de l'utilisateur. \\
				\hline
				lastName & String & O & Nom du participant. \\
				\hline
				firstName & String & O & Prénom du participant. \\
				\hline

			\end{tabularx}
		\end{center}
		
	\newpage
		\paragraph{Exemple JSON}
			\paragraph{}
			\begin{lstlisting}[language=json,firstnumber=1]
{
  id: 2
  day: "Mon"
  time: "1970/01/01 20:10"
  idTypeSession: [{
  	0: {
	  id: 55
	  IdPerson: {
	    id: 1,
	    LastName: "Isabelle",
	    FirstName: "Smits"
	    }
	  }
   }]
}
\end{lstlisting}
			
			
	\subsubsection{Code}
		\paragraph{}
			\href{https://github.com/victorsmits/Aquabike/blob/master/backend/src/Controller/API/ProfileControllerApi.php}{ProfileControllerApi.php}


\vspace{\baselineskip}
\section{/api/TypeSession}
	\paragraph{}
	Requête POST pour le renouvellement de l'abonnement d'un utilisateur.

\subsection{Requête}
	\subsubsection{paramètre du path}
		\paragraph{}
			Pas de Paramètre pour le path.
	
	\subsubsection{Content type}
		\paragraph{}
			application/json
			
	\subsubsection{Body}
		\paragraph{}
			\begin{center}
				\begin{tabularx}{\textwidth}{| T{0.2122\textwidth} | S{0.22\textwidth} | S{0.15\textwidth} | S{0.3\textwidth} |}
					\hline
					Propriété & Type & Obligatoire & Description \\
					\hline
					id & int & O & Id de l'utilisateur. \\
					\hline
				\end{tabularx}
			\end{center}
			
	\newpage
		\paragraph{Exemple JSON}
			\paragraph{}
			\begin{lstlisting}[language=json,firstnumber=1]
{
  id: 1
}
\end{lstlisting}

\subsection{Réponse}
	\subsubsection{Réussite}
		\paragraph{}
			Retourne code 200
			
	\subsubsection{Content-type}
		\paragraph{}
			application/json
	
	\subsubsection{Body}
		\begin{center}
			\begin{tabularx}{\textwidth}{| T{0.2122\textwidth} | S{0.22\textwidth} | S{0.15\textwidth} | S{0.3\textwidth} |}
				\hline
				Propriété & Type & Obligatoire & Description \\
				\hline
				result & Boolean & O & Status de la requête. \\
				\hline
			\end{tabularx}
		\end{center}
		
		\paragraph{Exemple JSON}
			\paragraph{}
			\begin{lstlisting}[language=json,firstnumber=1]
{
  result: true
}
\end{lstlisting}
			
			
	\subsubsection{Code}
		\paragraph{}
			\Href{https://github.com/victorsmits/Aquabike/blob/master/backend/src/Controller/API/AbonnementControllerApi.php}{AbonnementControllerApi.php}
