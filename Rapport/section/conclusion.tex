\section{Objectif atteints et non atteints}
	\subsection{Compatible Symfony-Twig et Symfony-Angular}
		\begin{center}
			\begin{tabularx}{\textwidth}{|Y|T{0.1\textwidth}|T{0.15\textwidth}|Y|}
				\hline
				User Stories & Priorité & Réalisation & Reste à faire \\
				\hline	
				En tant qu’utilisateur, je dois pouvoir me créer un compte sur le site & 2 & 100\% & \\
				\hline
				En tant qu'utilisateur, je dois pouvoir sélectionner le mois et l'année pour afficher les sessions qui m'intéresse & 2 & 80\% & Sélection de l'année dans la partie symfony-twig et affichage des message d'erreur sur le site Angular. \\
				\hline
				En tant qu’utilisateur, je dois pouvoir m’inscrire à une session un certain jour & 1 & 90\% & Afficher le message d'erreur sur le site Angular.\\
				\hline
				En tant qu’utilisateur, je dois pouvoir être inscrit automatiquement à une session & 1 & 80\% & Effectuer la boucle d'inscription pour la partie Symfony-Twig.  \\
				\hline
				En tant qu’utilisateur, je dois pouvoir me désinscrire à une session & 1 & 90\% & Afficher le message d'erreur sur le site Angular.\\
				\hline
				En tant qu’utilisateur, je dois pouvoir voir qui est inscrit pour chaque session & 2 & 100\% & \\
				\hline
			\end{tabularx}
		\end{center}
	
	\newpage
	\begin{center}
		\begin{tabularx}{\linewidth}{|Y|T{0.1\textwidth}|T{0.15\textwidth}|Y|}
			\hline
			En tant qu’utilisateur, je veux pouvoir voir dynamiquement le nombre de séance qui reste dans mon abonnement & 2 & 100\% &\\
			\hline
			En tant qu'utilisateur, je veux pouvoir modifier mon profile & 1 & 100\% &\\
			\hline
			En tant qu'utilisateur, je veux pouvoir m'inscrire a une session uniquement si il me reste des abonnements a disposition. & 2 & 100\% &\\
			\hline
			En tant qu’administrateur, je dois pouvoir annuler une session & 2 & 90\% & Afficher les messages d'erreur sur le site Angular. \\
			\hline
			En tant qu'administrateur, je dois pouvoir gérer les abonnement des utilisateurs & 2 & 100\% & \\
			\hline
			En tant qu'administrateur, je dois pouvoir créer une nouvelle session peut importe la date & 2 & 100\% &\\
			\hline

		\end{tabularx}
	\end{center}
	
	\newpage
	\subsection{Compatible Symfony-Angular uniquement}
		\begin{center}
			\begin{tabularx}{\linewidth}{|Y|T{0.1\textwidth}|T{0.15\textwidth}|Y|}
				\hline
				En tant qu'administrateur, je dois pouvoir générer les sessions pour un nombre d'année & 1 & 100\% & \\
				\hline
				En tant qu'administrateur, je dois pouvoir créer un type de session & 2 & 100\% & \\
				\hline
				En tant qu'administrateur, je veux pouvoir gérer les types de sessions disponible pour l'inscription automatique & 2 & 100\% &\\
				\hline
				En tant qu'administrateur, je veux pouvoir modifier un type de session & 1 & 100\% &\\
				\hline
				En tant qu'administrateur, je veux pouvoir supprimer un type de session & 1 & 90\% & Afficher le message d'erreur sur le site Angular.\\
				\hline
				En tant qu'administrateur, je veux pouvoir supprimer un utilisateur du système & 1 & 90\% & Afficher le message d'erreur sur le site Angular.\\
				\hline
			\end{tabularx}
		\end{center}
		
\section{Piste d'amélioration}
	\begin{itemize}
		\item Optimisation des requêtes de base de donnée.
		\item Nettoyage du code source (duplication, réorganisation, …).
		\item Amélioration/Finissions de la responsivité du site.
		\item Triage des sessions reçus dans angular pour un affichage trié en fonction du jour et de l'heure.
		\item Renomage des clés des json envoyer au frontend.
		\item Renomage de certain champs de la base de donnée.
		\item Utilisation d'une base de donnée orienté graph.
		\item Lors de la creation d'une session, trouver un type de session compatible si existant sinon mettre le paramètre à NULL. 
	\end{itemize}